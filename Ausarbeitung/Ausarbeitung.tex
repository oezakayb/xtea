% Diese Zeile bitte -nicht- aendern.
\documentclass[course=asp]{aspdoc}

%%%%%%%%%%%%%%%%%%%%%%%%%%%%%%%%%
%% TODO: Ersetzen Sie in den folgenden Zeilen die entsprechenden -Texte-
%% mit den richtigen Werten.
\newcommand{\theGroup}{IhreGruppennummer} % Beispiel: 42
\newcommand{\theNumber}{IhreProjektnummer} % Beispiel: A123
\author{Vorname1 Nachname1 \and Vorname2 Nachname2 \and Vorname3 Nachname3}
\date{AktuellesSemester} % Beispiel: Wintersemester 2019/20
%%%%%%%%%%%%%%%%%%%%%%%%%%%%%%%%%

% Diese Zeile bitte -nicht- aendern.
\title{Gruppe \theGroup{} -- Abgabe zu Aufgabe \theNumber}

\begin{document}
\maketitle

\section{Einleitung}


\section{Lösungsansatz}


% TODO: Je nach Aufgabenstellung einen der Begriffe wählen
\section{Korrektheit/Genauigkeit}


\section{Performanzanalyse}


\section{Zusammenfassung und Ausblick}

% TODO: Fuegen Sie Ihre Quellen der Datei Ausarbeitung.bib hinzu
% Referenzieren Sie diese dann mit \cite{}.
% Beispiel: CR2 ist ein Register der x86-Architektur~\cite{intel2017man}.
\bibliographystyle{plain}
\bibliography{Ausarbeitung}{}

\end{document}
